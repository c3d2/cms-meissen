\documentclass[12pt]{beamer}
%\documentclass[20pt,handout]{beamer}
\usetheme{Darmstadt}
\usepackage{graphicx}
\usepackage[german]{babel}
\usepackage[T1]{fontenc}
\usepackage[utf8]{inputenc}
\usepackage{tikz}
\setbeamertemplate{footline}[frame number]

\newcommand{\cc}[1]{\includegraphics[height=4mm]{img/#1.png}}
\usepackage{ifthen}
\newcommand{\license}[2][]{\\#2\ifthenelse{\equal{#1}{}}{}{\\\scriptsize\url{#1}}}
\usepackage{textcomp}

\pgfdeclareimage[height=.6cm]{c3d2logo}{./img/c3d2.pdf} 


\pgfdeclarelayer{foreground}
\pgfsetlayers{main,foreground}
\logo{\pgfputat{\pgfxy(-1,0)}{\pgfbox[center,base]{\pgfuseimage{c3d2logo}}}}


\title{NSA, Prism und co - Wie schützt man sich vor Überwachung?}
\author{\small Marius Melzer \& Stephan Thamm\\\large Chaos Computer Club Dresden}
\date{16.10.2013}

\begin{document}
\maketitle

\section{Einleitung}
\subsection{}

\begin{frame}
  \frametitle{Wer sind wir?}
  \begin{figure}
    \includegraphics[height=0.7\textheight]{img/fingerabdruck.jpg}
  \end{figure}
\end{frame}

\begin{frame}
  \frametitle{Wer sind wir?}
  \begin{figure}
    \includegraphics[height=0.7\textheight]{img/trojaner.jpg}
  \end{figure}
\end{frame}

\begin{frame}
    \frametitle{Wer sind wir?}
    \begin{itemize}
      \item<1-> Chaos Computer Club Dresden (\url{http://c3d2.de})
          \note{}
      \item<2-> Datenspuren: Herbst 2014 \url{http://datenspuren.de}
      \item<3-> Podcasts (\url{http://pentamedia.de})
      \item<4-> Chaos macht Schule
    \end{itemize}
\end{frame}

\begin{frame}
    \frametitle{Agenda}
    \begin{itemize}
      \item Vertraulichkeit
      \item Integrität
      \item Anonymität
    \end{itemize}
\end{frame}

\section{Vertraulichkeit}
\subsection{}

\section{Integrität}
\subsection{}

\begin{frame}
    \frametitle{Infrastruktur}
    Wahrung der Integrität des Netzes möglich durch:
    \begin{itemize}
      \item<1-> Netzneutralität
      \item<2-> Dezentrale, internationale und transparente Aufsicht über Infrastruktur
      \item<3-> Abkommen zu Cyberwar und integrität fremder Netze
    \end{itemize}
\end{frame}

\begin{frame}
    \frametitle{Zensur}
    \includegraphics[height=0.7\textheight]{img/zensur-guardian.jpg}
\end{frame}

\begin{frame}
    \frametitle{Arten von Zensur}
    \begin{itemize}
      \item<1-> DNS-basiert
      \item<2-> IP-basiert
      \item<3-> Inhaltsbasiert
    \end{itemize}
\end{frame}

\begin{frame}
    \frametitle{TOR}
    \includegraphics[height=0.7\textheight]{img/tor1.png}
\end{frame}

\begin{frame}
    \frametitle{TOR}
    \includegraphics[height=0.7\textheight]{img/tor2.png}
\end{frame}

\begin{frame}
    \frametitle{TOR}
    \includegraphics[height=0.7\textheight]{img/tor3.png}
\end{frame}

\begin{frame}
    \frametitle{TOR}
    \includegraphics[height=0.7\textheight]{img/vidalia.png}
\end{frame}

\begin{frame}
    \frametitle{TOR}
    \begin{center} \Large Bridges \end{center}
\end{frame}

\begin{frame}
    \frametitle{TOR}
    \includegraphics[height=0.7\textheight]{img/bridge1.png}
\end{frame}

\begin{frame}
    \frametitle{TOR}
    \includegraphics[height=0.7\textheight]{img/bridge2.png}
\end{frame}

\begin{frame}
    \frametitle{Collusion}
    \begin{center} \Large Collusion \end{center}
\end{frame}

\begin{frame}
    \frametitle{Gegenmaßnahmen}
    \begin{center} \Large Pseudonymität \end{center}
\end{frame}

\begin{frame}
    \frametitle{Gegenmaßnahmen}
    \begin{center} \Large Pseudonymität \end{center}
\end{frame}


\end{document}
